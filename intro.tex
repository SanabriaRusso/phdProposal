Carrier Sense Multiple Access with Collision Avoidance (CSMA/CA) is the protocol used in Wireless Local Area Networks (WLANs) to coordinate transmissions. Nodes should avoid simultaneous transmissions because the medium is shared, so concurrent transmissions attempts will result in indecipherable messages to the receivers. This event is referred to as a \emph{collision}. 

For CSMA/CA, time is slotted. As a result, there are three kind of slots: \emph{empty}, \emph{successful} and \emph{colllision} slots, where successful and collision slots contain successful transmissions or collision events. While the remaining are just tiny empty slots of a fixed time length.

Every time there is a contention for transmission, CSMA/CA forces contenders to count down from a randomly generated number (from now on referred to as backoff counter), decrementing it by one per every passing empty slot. When the backoff expires (reaches zero), contenders will attempt transmission. Nevertheless, because the backoff counter is generated at random, there might be cases where two or more contenders simultaneously attempt transmission and a collision occurs, significantly degrading the throughput of the system.

It is possible to obtain greater levels of throughput than the achieved by CSMA/CA under optimal parameter configuration by picking a deterministic backoff counter after successful transmissions. This approach is called Carrier Sense Multiple Access with Enhanced Collision Avoidance (CSMA/ECA)~\cite{CSMA_ECA}. Results also show that by making simple modifications on the behavior of the current protocol, CSMA/ECA is able to allocate more contenders in a collision-free fashion while preserving the system fairness by equally distributing the system throughput among all nodes. 
% Furthermore, CSMA/ECA is resilient to synchronization flaws on the wireless network cards that can cause a miscount of passing slots (slot drift), as opposed to other MAC protocols~\cite{slotDrift}.
% 

Many years of testing have settled CSMA/CA as the default protocol for this type of networks, even-though many other proposals claim to outperform it, e.g.~\cite{CSMA_ECA,bellalta2009vtc,L_MAC2,HE,fairness-ECA}. Nevertheless, their proposed adjustments tested by simulation are not included in the current standard.

Recent approaches to design and implement MAC protocols on cheap commodity hardware~\cite{WMP,FLAVIA} opened the possibility to prototype some of the protocols proposed by the research community. Although at an early phase and steep learning curve, these alternatives allow researchers of all levels to make substantial contributions.

\subsection{Motivation}\label{motivation}
As mentioned before, many proposals to amend the collision problem in CSMA/CA have been made and none is included in the current standard.

Taking a guess-look at what is to come in a few years time, WLANs are expected to be as crowded as never before. From tablets, laptops, smart phones, watches, smart health/activity monitoring devices; to traffic prioritization, accommodating these many devices and services will soon out-challenge CSMA/CA.

Even-though CSMA/CA in theory is able to coordinate medium access for many contenders, it does so at the price of a reduced throughput induced by collisions. This is completely leveraged by CSMA/ECA, which in fact provides greater throughput than CSMA/CA in almost every testable scenario.

The goal of this PhD Thesis is twofold: $1$) further analyze CSMA/ECA behavior considering unsaturated scenarios, and $2$) to write the protocol into cheap commodity Wireless Network Interface Cards (WNICs) using the principles proposed in \cite{WMP}. The expected results from this work are the evaluation and testing in real hardware of a MAC protocol capable of allocating a large number of contenders in a collision-free fashion, ensuring long-term throughput fairness and able to comply with current quality of service (QoS) specifications for WLANs. 

This work is expected to collect enough evidence for the community to consider its contributions for upcoming amends to the standard. Furthermore, the progress reports and tools developed during the research effort will provide sufficient background for its replication and further development of MAC protocols in real hardware.



% The expected results from this work will provide a complete overview of CSMA/ECA and sufficient documentation to make an amendment proposal to CSMA/CA.