\documentclass[]{llncs}

\usepackage{amsmath}

\begin{document}

\title{Collision-Free WLANs: From Concepts to Working Protocols. A PhD. Proposal}
\author{Luis Sanabria-Russo}
\institute{Universitat Pompeu Fabra, Barcelona Spain \\ \email{luis.sanabria@upf.edu}}
\maketitle

\begin{abstract}
The current standard for medium access control in Wireless Local Area Networks (WLANs) called Carrier Sense Multiple Access with Collision Avoidance (CSMA/CA), is by its nature prone to collisions. These happen as a consequence of randomizing the backoff counter each contender must select in order to coordinate its transmissions. Carrier Sense Multiple Access with Enhanced Collision Avoidance (CSMA/ECA) introduces minor adjustments to CSMA/CA  that eliminate the possibility of collisions after a short convergence period. Its principle is to select a deterministic backoff after successful transmissions instead of a random one (as in CSMA/CA), allowing contenders to \emph{own} a time slot in the schedule. Simulation results show an important increase in the offered throughput as well as the number of contenders that can me allocated in a collision-free fashion.
\end{abstract}

\section{Introduction}\label{introduction}
	Carrier Sense Multiple Access with Collision Avoidance (CSMA/CA) is the protocol used in wireless local area networks (WLANs) to coordinate transmissions. Nodes should avoid simultaneous transmissions because the medium is shared, so concurrent transmissions attempts will result in indecipherable messages to the receivers. This event is referred to as a \emph{collision}. 

For CSMA/CA, time is slotted. As a result, there are three kind of slots: \emph{empty}, \emph{successful} and \emph{colllision} slots, where successful and collision slots contain succesful transmissions or collision events. While the remaining are just tiny empty slots of a fixed time length.

Every time there is a contend for transmission, CSMA/CA forces contenders to count down from a randomly generated number (from now on referred to as backoff counter), decrementing it by one per every passing empty slot. When the backoff expires (reaches zero), contenders will attempt transmission. Nevertheless, because the backoff counter is generated at random, there might be cases where two o more contenders simultaneuously attempt transmission and a collision occurs, significantly degrading the throughput of the system as more nodes join the contend for the medium.

The focus of this paper is to describe how it is possible to obtain greater levels of throughput than the achieved by CSMA/CA under optimal parameter configuration, by means of picking a deterministic backoff counter after successful transmissions. This approach is called Carrier Sense Multiple Access with Enhanced Collision Avoidance (CSMA/ECA)~\cite{CSMA_ECA}. Results also show that by making this simple modification on the behavior of CSMA/CA, CSMA/ECA preserves the system fairness by equally distributing the system throughput among all contenders. Furthermore, CSMA/ECA is resilient to syncronization flaws on the wireless network cards that can cause a misscount of passing slots (slot drift), as opposed to other MAC protocols~\cite{slotDrift}.


\bibliographystyle{splncs}
\bibliography{ref,my_bib}

\end{document}

