As was mentioned in the previous sections, picking a deterministic backoff after successful transmissions provides higher throughput than CSMA/CA in most common scenarios. CSMA/ECA's backoff strategy allows nodes to achieve a collision-free state even for a very large number of CSMA/ECA contenders; while its fairness mechanisms ensure that all stations share the same available system throughput in the long-term.

The following paragraphs detail the general and specific objectives of this research. 

\subsection{General Objective}
% In Section~\ref{motivation} is stated that the goal of this PhD Thesis is twofold. Nevertheless, its sole general objective can be described as to:
\begin{itemize}
	\item Investigate and develop methods that would allow the construction of a collision-free MAC protocol capable of coping with upcoming Small-Office/Home-Office (SOHO) scenarios, while facilitating its prototyping on real hardware.
	
% 	Design methods and procedures allowing the construction and prototyping in real hardware of a totally-distributed collision-free MAC protocol, capable of allocating the high number of contenders expected to be seen in upcoming Small-Office/Hom-Office (SOHO) scenarios.

% 	\item Design and implement in real hardware a totally distributed collision-free MAC protocol for 802.11-like networks, capable of allocating the high number of contenders expected to be seen in upcoming Small-Office/Home-Office (SOHO) scenarios and ensuring greater levels of throughput than the current standard.
\end{itemize}

\subsection{Specific Objectives}
Ranging from research challenges to debugging, the following subsection details what are believed to be the required steps to accomplish this research's general objective. Each specific objective represent a result of it own, while their \emph{phases} dictate the activities required to fulfill it.
\begin{enumerate}

	\item Investigate and develop mechanisms to allocate a great number of contenders in a collision-free fashion while considering Quality of Service (QoS) requirements of IEEE 802.11e networks\\
% 	\item Design mechanisms for CSMA/ECA to work with upcoming traffic and Quality of Service (QoS) requirements in WLANs and evaluate its performance.\\
	
	{\bfseries Phases:}
	\begin{enumerate}
		\item Investigate and develop mechanisms that would allow CSMA/ECA to accommodate many contenders in a collision-free fashion.\label{ECAHysteresis}
		\item Design and evaluate the operation of CSMA/ECA in conjunction with the Enhanced Distribution Channel Access (EDCA).
		\item Develop CSMA/ECA in a discrete event-based simulator.\label{incorporateECA}
		\item Identify the performance metrics that will allow an unbiased comparison with CSMA/CA.\label{metrics}
		\item Design simulation tests for metrics gathering under different traffic patterns, QoS settings and considering mixed scenarios (shared network with some nodes running CSMA/CA and others CSMA/ECA).\label{scenarios}
		\item Perform the simulation tests designed in the previous phase.
		\item Document the results for future comparison with the hardware implementation.\label{simulationResults}\\
	\end{enumerate}
	{\bfseries Result:} Performance evaluation of CSMA/ECA under EDCA.\\


% 	\item Compare CSMA/CA performance with CSMA/ECA by means of simulation.\\
% 	
% 	{\bfseries Phases:}
% 	\begin{enumerate}
% 		\item Replicate CSMA/CA's behavior in a discrete event-based simulator.\label{buildSimulator}
% 		\item Progressively incorporate CSMA/ECA features.\label{incorporateECA}
% 		\item Identify the performance metrics that will allow an unbiased comparison between the two protocols.\label{metrics}
% 		\item Perform simulation tests for metrics gathering under saturated and unsaturated scenarios\label{scenarios}.
% 		\item Construct a mixed scenario, where some nodes run CSMA/CA while of others use CSMA/ECA. Repeat the simulations as in Phase~\ref{scenarios}.
% 		\item Document the results for future comparison with the hardware implementation.\label{simulationResults}\\
% 	\end{enumerate}
	
	\item Design methods to develop and test CSMA/CA into the WMP platform.\label{learningWMP}\\
	
	{\bfseries Phases:}
	\begin{enumerate}
		\item Install the WMP-Editor on a host PC to study its components and configuration.\label{installWMP}
		\item Identify the software and hardware requirements needed to replicate CSMA/CA as a WMP.\label{WMPRequirements}
		\item Design methods to develop CSMA/CA as a WMP.\label{WMPCSMA-CA}
		\item Develop a WMP containing the CSMA/CA functionality (WMP-CSMA/CA), ready to be uploaded to the appropriate WNICs.
		\item Design evaluation methods for the WMP-CSMA/CA so its performance can be compared with analytical models.\label{evaluatingWMP}
		\item Develop automated metric gathering scripts on the client PCs in order to achieve the goals set at Phase~\ref{evaluatingWMP}.
		\item Upload the WMP-CSMA/CA into the necessary WNICs.
		\item Perform the evaluations designed in Phase~\ref{evaluatingWMP} and analyze the results.\label{WMPExperiment}\\
	\end{enumerate}
	{\bfseries Result:} Method, procedures and performance evaluation of CSMA/CA as a running WMP.\\

	\item Progressively modify WMP-CSMA/CA into WMP-CSMA/ECA.\label{ECAinWMP}\\
	
	{\bfseries Phases:}
	\begin{enumerate}
		\item Identify the required modifications in WMP-CSMA/CA needed for it to become WMP-CSMA/ECA.\label{WMPModifications}
		\item Design methods for translating the missing characteristics into functioning XFSM.\label{accessByteCode}
		\item Develop CSMA/ECA into a WMP.
		\item Design evaluation methods for WMP-CSMA/ECA so its performance can be compared against the results obtained in Phase~\ref{simulationResults}.
		\item Perform the evaluations designed in the previous phase.\\
	\end{enumerate}

	\item Extend the functionality of CSMA/ECA to RFID in the attempt to reduce convergence and tag-reading time.\label{ECAinRFID}\\
	
	{\bfseries Phases:}
	\begin{enumerate}
		\item Gather metrics about the current performance of the system.
		\item Identify the required modifications to CSMA/ECA so it can work in a RFID environment.
		\item Design methods for introducing the required modifications to CSMA/ECA.
		\item Design simulation scenarios so the modified CSMA/ECA's performance can be analyzed.
		\item Perform the simulations designed in the previous phase.
		\item Document the results and compare them with the performance of the network before incorporating CSMA/ECA.\\
	\end{enumerate}

\end{enumerate}
