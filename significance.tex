It is not hard to encounter the crowded WLANs scenario exemplified in Section~\ref{motivation}. Even nowadays it is not rare to have multiple WiFi devices at home attempting to access the channel at the same time: watching an on-line video stream, surfing the Web, receiving VoIP calls and uploading data from your personal health monitoring device.

CSMA/CA has been the de facto standard for coordinating transmissions in WLANs; nevertheless, its performance degrades when imposing heavy traffic on crowded scenarios like the one proposed above, where tens of devices must coordinate their transmission attempts in a totally distributed manner. This degradation in the system performance can be appreciated in the form of video lags while streaming or in below-average download speeds.

By removing the randomness from the contention mechanism of CSMA/CA, CSMA/ECA achieves a better performance while allowing many nodes to coexist in a collision-free environment. Nevertheless, CSMA/ECA lacks a thorough study of the different scenarios that it could face regarding: Quality of Service (QoS), hidden/exposed node mechanisms, performance against different traffic patterns and contention parameters optimization; which on the other hand have been the focus of study in CSMA/CA for many years.

Although it has proven to best CSMA/CA's throughput in controlled simulation environments, CSMA/ECA is far away from being considered as an amend to the standard.

This work aims at bringing the benefits of CSMA/ECA to home WLANs. 